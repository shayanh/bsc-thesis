% !TeX root=../main.tex
\chapter{روش تحقیق}
%\thispagestyle{empty} 
\section{مقدمه} 
در این فصل نخست، روش پیشنهادی خود را برای درستی‌سنجی نرم‌افزارهای با معماری میکروسرویس ارائه می‌دهیم. سپس به معرفی و بررسی پیاده‌سازی انجام‌شده خواهیم پرداخت.

\section{روش پیشنهادی}

\begin{algorithm}[ht]
\onehalfspacing
\caption{الگوریتم میان‌افزار سمت سرور} \label{alg:serverMiddleware}
\begin{latin}
\begin{algorithmic}[1]

\REQUIRE $Request$, $RPC$, $Contracts$, $CallHistory$
\STATE $C \gets null$
\FORALL{$R$ such that $R \in Contracts$}
    \IF{$R.RPC = RPC$}
        \STATE $C \gets R$
    \ENDIF
\ENDFOR
\IF{$C \neq null$}
    \FORALL{$PRE$ such that $PRE \in C.PreConditions$}
        \IF {$PRE(Request) \neq true$}
            \STATE log failed precondition
        \ENDIF
    \ENDFOR
\ENDIF
\STATE $Response \gets RPC(Request)$
\IF{$C \neq null$}
    \FORALL{$POST$ such that $POST \in C.PostConditions$}
        \IF {$POST(Response, Request, CallHistory[Request.ID]) \neq true$}
            \STATE log failed postcondition
        \ENDIF
    \ENDFOR
\ENDIF

\end{algorithmic}
\end{latin}
\end{algorithm}

\begin{algorithm}[ht]
\onehalfspacing
\caption{الگوریتم میان‌افزار سمت کلاینت} \label{alg:clientMiddleware}
\begin{latin}
\begin{algorithmic}[1]

\REQUIRE $Request$, $RPC$, $CallHistory$
\STATE $Response \gets RPC(Request)$
\STATE $CallHistory[Request.ID]$.append($RPC, Response, Request$)

\end{algorithmic}
\end{latin}
\end{algorithm}

با بزرگ شدن اندازه برنامه و بیشتر شدن پیچیدگی‌های آن، اطمینان از درستی این برنامه‌های مبتنی بر معماری میکروسرویس برای گردانندگان آن سخت‌تر می‌شود. ویژگی‌های این برنامه‌ها، وابسته است به رفتار هر کدام از میکروسرویس‌های تشکیل دهنده آن، و تنظیمات مربوط به ارتباطات بین آن‌ها. فهمیدن برقراری یک شرط خاص در هنگام اجرای یک برنامه یک مساله غیربدیهی برای برنامه‌های کوچک است، اما برای برنامه بزرگی که از تعدادی زیاد میکروسرویس تشکیل شده است، بسیار سخت است.

در این تحقیق ما یک روش برای درستی‌سنجی برنامه‌های با معماری میکروسرویس ارائه می‌کنیم به گونه‌ای که مشکلات تحقیقات قبلی را نداشته باشد. یعنی امکان ارائه پیاده‌سازی مناسب محیط‌های عملیاتی برای آن عملی باشد؛ و چارچوبی ارائه دهد تا توسعه‌دهندگان به راحتی بتوانند شرط‌های مورد نظر خود را توصیف کنند، چرا که در غیر این صورت در عمل نمی‌توان از روش پیشنهاد شده استفاده کرد.

در این تحقیق ما روی درستی‌سنجی ارتباط بین میکروسرویس‌ها تمرکز می‌کنیم چرا که پیچیدگی اصلی در در این معماری، ارتباطات زیاد بین تعداد زیادی سرویس است که روی شبکه و با استفاده از
\gls{RPC}
با هم صحبت می‌کنند.

با توجه به اهداف گفته شده، ما روش طراحی بر اساس قرارداد را برای نوشتن شرط‌ها و بررسی آن‌ها انتخاب کردیم. طراحی بر اساس قرارداد روش قدرتمندی است که با استفاده از آن به خوبی می‌توان پیش‌شرط‌ها و پس‌شرط‌های لازم را برای هر
\gls{RemoteProcedureCall}
نوشت. ما این شرط‌ها در زمان اجرای برنامه بررسی می‌کنیم و در صورت نقض شدن هر شرط، آن را به توسعه‌دهنده اطلاع می‌دهیم. در این صورت، اشکالات در کل سیستم پخش نمی‌شوند و توسعه‌دهنده در اولین فرصت متوجه آن‌ها می‌شود.

در روش ما این امکان وجود دارد که بتوان روی ترتیب اجرا
\gls{RPC}ها
شرط نوشت. این موضوع کمک می‌کند که بسیار بهتر بتوان ارتباط میان میکروسرویس‌ها را درستی‌سنجی کرد. در قسمت بعدی به طور دقیق‌تر تعریف شرط‌ها را بررسی می‌کنیم. 

ما برای بررسی قراردادها، یک
\gls{Middleware}
برای سرور و یک میان‌افزار دیگر برای کلاینت ارائه می‌کنیم. میان‌افزار سمت سرور برقراری پیش‌شرط‌ها و پس‌شرط‌ها را بررسی می‌کند. نحوه کار این میان‌افزار در الگوریتم
\ref{alg:serverMiddleware}
آمده است.

میان‌افزار سمت کلاینت، فراخوانی‌های از راه‌ دور را دنبال می‌کند تا بتوان در پس‌شرط‌ها از آن‌ها استفاده کرد. نحوه کار این میان‌افزار در الگوریتم
\ref{alg:clientMiddleware}
آمده است.


\section{پیاده‌سازی}

ما برای روش خود، یک پیاده‌سازی در زبان برنامه‌نویسی
Go
و برای چارچوب
gRPC
ارائه انجام داده‌ایم که این برنامه به صورت متن‌باز
\cite{grpcGoContracts}
در دسترس است. ما برای بررسی قراردادها، یک
\gls{Middleware}
برای سرور و یک میان‌افزار دیگر برای کلاینت gRPC ارائه می‌کنیم. میان‌افزار سمت سرور برقراری پیش‌شرط‌ها و پس‌شرط‌ها را بررسی می‌کند و میان‌افزار سمت کلاینت، فراخوانی‌های از راه‌ دور را دنبال می‌کند تا بتوان در پس‌شرط‌ها از آن‌ها استفاده کرد.

برای بررسی دقیق‌تر این روش، یک مثال را در نظر می‌گیریم. فرض کنید که سه سرویس تعریف شده در زبان توصیف 
\lr{Protocol Buffers}
در برنامه 
\ref{code:SampleProto}
را برای یک فروشگاه آنلاین داریم.

\singlespacing
\begin{figure}
	\begin{LTR}
		\lstinputlisting[language=protobuf2, breaklines=true, basicstyle=\ttfamily\footnotesize, caption={سرویس‌های فروشگاه آنلاین}, label={code:SampleProto}]{sample.proto}
	\end{LTR}
\end{figure}
\doublespacing

ما می‌خواهیم سرویس 
\lr{\texttt{CheckoutService}}
را درستی‌سنجی کنیم. این سرویس تنها یک فراخوانی از راه‌ دور
\lr{\texttt{PlaceOrder}}
را دارد و می‌خواهیم برای آن پیش‌شرط‌ها و پس‌شرط‌هایی را بنویسیم. این تابع به این صورت عمل می‌کند که ابتدا هزینه کل سفارش را محاسبه می‌کند، سپس عملیات پرداخت را از طریق سرویس
\lr{\texttt{PaymentService}}
انجام می‌دهد. در صورت موفقیت عملیات پرداخت، می‌بایست عملیات ارسال کالا از طریق سرویس
\lr{\texttt{ShippingService}}
انجام شود و صورت موفق آمیز بودن، عملیات ثبت سفارش با موفقیت انجام می‌شود. پیاده‌سازی ساده‌شده تابع
\lr{\texttt{PlaceOrder}}
در برنامه
\ref{code:PlaceOrder}
آمده است.

\singlespacing
\begin{figure}
	\begin{LTR}
		\lstinputlisting[language=GO, caption={
		پیاده‌سازی ساده‌شده تابع
		\lr{\texttt{PlaceOrder}}
		در زبان Go
		}, label={code:PlaceOrder}, breaklines=true, basicstyle=\ttfamily\footnotesize]{placeorder.go}
	\end{LTR}
\end{figure}
\doublespacing


ما می‌خواهیم که دو پیش‌شرط زیر را برای این تابع بنویسیم:

\begin{enumerate}
\item
احراز هویت کاربر باید تایید شده باشد. به همین جهت فیلد
\lr{\texttt{user\_id}}
درخواست نباید خالی باشد.

\item
ایمیل کاربر باید معتبر باشد.
\end{enumerate}


و از برقراری پس‌شرط‌های زیر هم می‌خواهیم اطمینان حاصل کنیم:
\begin{enumerate}
\item
در صورت موفقیت‌آمیز بودن خروجی تابع
\lr{\texttt{PlaceOrder}}
، باید تابع
\lr{\texttt{Charge}}
در
\lr{\texttt{PaymentService}}
با موفقیت فراخوانی شده باشد.
\item
تابع
\lr{\texttt{Charge}}
باید قبل از تابع
\lr{\texttt{ShipOrder}}
فراخوانی شده باشد.
\end{enumerate}


برای نوشتن شرط‌ها با استفاده از کتابخانه
\lr{go-grpc-contracts}
باید با دو مفهوم اصلی این کتابخانه آشنا شویم. در ابتدا یک
\lr{\texttt{ServerContract}}
برای یک سرور gRPC تعریف می‌شود. این شی شامل قراردادهایی است که برای یک سرور تعریف می‌شود. سپس
\lr{\texttt{UnaryRPCContract}}
تعریف می‌شود. این شی شامل قراردادهایی است که برای یک RPC خاص تعریف می‌شود. سپس می‌بایست قراردادهای مربوط به هر
\lr{RPC}،
در قرارداد سرور مربوط ثبت شوند. در برنامه 
\ref{code:Register}
نحوه تعریف قراردادها و چگونگی استفاده از میان‌افزار مشخص شده است.

\singlespacing
\begin{figure}
	\begin{LTR}
		\lstinputlisting[language=GO, caption={
		تعریف قرارداد‌ها
		}, label={code:Register}, breaklines=true, basicstyle=\ttfamily\footnotesize]{register.go}
	\end{LTR}
\end{figure}
\doublespacing

برای تعریف پیش‌شرط‌ها، باید یک تابع که ورودی آن درخواست ورودی
\lr{RPC}
است نوشت، و آن‌ را به عنوان 
\lr{\texttt{PreCondition}}
در
\lr{\texttt{UnaryRPCContract}}
ثبت کرد. در برنامه
\ref{code:Preconditions}
دو پیش‌شرط گفته شده پیاده‌سازی شده‌اند.

\singlespacing
\begin{figure}
	\begin{LTR}
		\lstinputlisting[language=GO, caption={
		پیش‌شرط‌های تابع
		\lr{\texttt{PlaceOrder}}
		}, label={code:Preconditions}, breaklines=true, basicstyle=\ttfamily\footnotesize]{preconditions.go}
	\end{LTR}
\end{figure}
\doublespacing

برای تعریف پس‌شرط‌ها، باید یک تابع که ورودی‌‌های آن خروجی
\lr{RPC}،
ورودی
\lr{RPC}
و یک شی از نوع 
\lr{\texttt{RPCCallHistory}}
است، نوشت. کلاس 
\lr{\texttt{RPCCallHistory}}
امکان دسترسی به RPCهای فراخوانی شده را می‌دهد. در برنامه
\ref{code:Postconditions}
پس‌شرط‌های گفته شده پیاده‌سازی شده‌اند.

\singlespacing
\begin{figure}
	\begin{LTR}
		\lstinputlisting[language=GO, caption={
		پس‌شرط‌های تابع
		\lr{\texttt{PlaceOrder}}
		}, label={code:Postconditions}, breaklines=true, basicstyle=\ttfamily\footnotesize]{postconditions.go}
	\end{LTR}
\end{figure}
\doublespacing
