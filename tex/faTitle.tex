% !TeX root=../main.tex
% در این فایل، عنوان پایان‌نامه، مشخصات خود، متن تقدیمی‌، ستایش، سپاس‌گزاری و چکیده پایان‌نامه را به فارسی، وارد کنید.
% توجه داشته باشید که جدول حاوی مشخصات پروژه/پایان‌نامه/رساله و همچنین، مشخصات داخل آن، به طور خودکار، درج می‌شود.
%%%%%%%%%%%%%%%%%%%%%%%%%%%%%%%%%%%%
% دانشگاه خود را وارد کنید
\university{دانشگاه تهران}
% پردیس دانشگاهی خود را اگر نیاز است وارد کنید (مثال: فنی، علوم پایه، علوم انسانی و ...)
\college{پردیس دانشکده‌های فنی}
% دانشکده، آموزشکده و یا پژوهشکده  خود را وارد کنید
\faculty{دانشکده مهندسی برق و کامپیوتر}
% گروه آموزشی خود را وارد کنید (در صورت نیاز)
\department{گروه مهندسی نرم‌افزار}
% رشته تحصیلی خود را وارد کنید
\subject{مهندسی کامپیوتر}
% گرایش خود را وارد کنید
\field{مهندسی نرم‌افزار}
% عنوان پایان‌نامه را وارد کنید
\title{درستی‌سنجی نرم‌افزارهای با معماری میکروسرویس}
% نام استاد(ان) راهنما را وارد کنید
\firstsupervisor{دکتر حسین حجت}
\firstsupervisorrank{استاد}
%\secondsupervisor{دکتر راهنمای دوم}
%\secondsupervisorrank{استادیار}
% نام استاد(دان) مشاور را وارد کنید. چنانچه استاد مشاور ندارید، دستورات پایین را غیرفعال کنید.
%\firstadvisor{دکتر مشاور اول}
%\firstadvisorrank{استادیار}
%\secondadvisor{دکتر مشاور دوم}
% نام داوران داخلی و خارجی خود را وارد نمایید.
%\internaljudge{دکتر داور داخلی}
%\internaljudgerank{دانشیار}
%\externaljudge{دکتر داور خارجی}
%\externaljudgerank{دانشیار}
%\externaljudgeuniversity{دانشگاه داور خارجی}
% نام نماینده کمیته تحصیلات تکمیلی در دانشکده \ گروه
%\graduatedeputy{دکتر نماینده}
%\graduatedeputyrank{دانشیار}
% نام دانشجو را وارد کنید
\name{شایان}
% نام خانوادگی دانشجو را وارد کنید
\surname{حسینی}
% شماره دانشجویی دانشجو را وارد کنید
\studentID{810194540}
% تاریخ پایان‌نامه را وارد کنید
\thesisdate{شهریور ۱۳۹۹}
% به صورت پیش‌فرض برای پایان‌نامه‌های کارشناسی تا دکترا به ترتیب از عبارات «پروژه»، «پایان‌نامه» و «رساله» استفاده می‌شود؛ اگر  نمی‌پسندید هر عنوانی را که مایلید در دستور زیر قرار داده و آنرا از حالت توضیح خارج کنید.
%\projectLabel{پایان‌نامه}

% به صورت پیش‌فرض برای عناوین مقاطع تحصیلی کارشناسی تا دکترا به ترتیب از عبارت «کارشناسی»، «کارشناسی ارشد» و «دکتری» استفاده می‌شود؛ اگر نمی‌پسندید هر عنوانی را که مایلید در دستور زیر قرار داده و آنرا از حالت توضیح خارج کنید.
%\degree{}
%%%%%%%%%%%%%%%%%%%%%%%%%%%%%%%%%%%%%%%%%%%%%%%%%%%%
%% پایان‌نامه خود را تقدیم کنید! %%
%\dedication
%{
%{\Large تقدیم به:}\\
%\begin{flushleft}{
%	\huge
%	همسر و فرزندانم\\
%	\vspace{7mm}
%	و\\
%	\vspace{7mm}
%	پدر و مادرم
%}
%\end{flushleft}
%}
%% متن قدردانی %%
%% ترجیحا با توجه به ذوق و سلیقه خود متن قدردانی را تغییر دهید.
\acknowledgement{
سپاس خداوندگار حکیم را که با لطف بی‌کران خود، آدمی را به زیور عقل آراست.

در آغاز وظیفه‌  خود  می‌دانم از زحمات بی‌دریغ اساتید  راهنمای خود،  جناب آقای دکتر حجت، صمیمانه تشکر و  قدردانی کنم که در طول انجام این پایان‌نامه با نهایت صبوری همواره راهنما و مشوق من بودند و قطعاً بدون راهنمایی‌های ارزنده‌ ایشان، این مجموعه به انجام نمی‌رسید.

%از همکاری و مساعدت‌های دکتر ... مسئول تحصیلات تکمیلی و سایر کارکنان دانشکده بویژه سرکار خانم ... کمال تشکر را دارم.

%با سپاس بی‌دریغ خدمت دوستان گران‌مایه‌ام، خانم‌ها ... و آقایان ... در آزمایشگاه ...، که با همفکری مرا صمیمانه و مشفقانه یاری داده‌اند.

و در پایان، تشکر می‌کنم از خانواده عزیزم که پشتیبان من بودند.
}
%%%%%%%%%%%%%%%%%%%%%%%%%%%%%%%%%%%%
%چکیده پایان‌نامه را وارد کنید
\fa-abstract{
امروزه بسیاری از برنامه‌های بزرگ توسط تعداد زیادی از میکروسرویس‌ها ساخته می‌شوند. هر کدام از این میکروسرویس‌ها در محیطی از بقیه مستقر می‌شود و با دیگر میکروسرویس‌ها از طریق
\gls{RemoteProcedureCall}
ارتباط برقرار می‌کند. استفاده از میکروسرویس‌ها سبب بهبود مقیاس‌پذیری - هر کدام از اجزای یک برنامه می‌توانند به طور مستقل رشد کنند - و استقرار می‌شود. اما این برنامه‌ به طور ذاتی توزیع‌شده هستند و ابزارهای فعلی این قابلیت را ندارند که بتوان درباره این برنامه‌ها استدلال کرد و یا از رفتارهای آن‌ها به طور کلی اطمینان حاصل کرد.
در این پروژه ما سیستمی را معرفی می‌کنیم تا با استفاده از آن بتوان ارتباط میان میکروسرویس‌ها درستی‌سنجی کرد. با الهام گرفتن از
\gls{DesignByContract}
، سیستم ما سرویس‌ها را در هنگام اجرا دیده‌بانی می‌کند تا سرویس‌هایی را که به قرارداد خود پایبند نیستند را پیدا کند. در نهایت، ما این سیستم را روی یک برنامه با معماری میکروسرویس پیاده می‌کنیم و نشان می‌دهیم که چگونه ایراداتی را در این برنامه پیدا کردیم بدون اینکه کارایی برنامه کاهش پیدا کند.

}
% کلمات کلیدی پایان‌نامه را وارد کنید
\keywords{درستی‌سنجی، میکروسرویس، قرارداد}
% انتهای وارد کردن فیلد‌ها
%%%%%%%%%%%%%%%%%%%%%%%%%%%%%%%%%%%%%%%%%%%%%%%%%%%%%%
