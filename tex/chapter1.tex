% !TeX root=../main.tex

\chapter{مقدمه و بیان مساله}
% دستور زیر باعث عدم‌نمایش شماره صفحه در اولین صفحه‌ی این فصل می‌شود.
%\thispagestyle{empty}
\section{مقدمه}
حروف‌چینی پروژه کارشناسی، پایان‌نامه یا رساله یکی از موارد پرکاربرد استفاده از
\lr{\LaTeX}
و زی‌پرشین
\cite{Khalighi87xepersian}
است. یک پروژه، پایان‌نامه یا رساله، احتیاج به تنظیمات زیادی از نظر صفحه‌آرایی دارد که وقت زیادی از دانشجو می‌گیرد. به دلیل قابلیت‌های بسیار لاتک در حروف‌چینی، کلاسی با نام 
\lr{tehran-thesis}
برای حروف‌چینی پروژه‌ها، پایان‌نامه‌ها و رساله‌های دانشگاه تهران، بر مبنای کلاس مشابه
\lr{IUST-Thesis}
تهیه شده است. این کلاس و فایل‌های همراه آن به گونه‌ای طراحی شده است که مطابق با دستورالعمل نگارش و تدوین پایان‌نامه کارشناسی ارشد و دکتری پردیس دانشکده‌های فنی دانشگاه تهران
\cite{UTThesisGuide}
باشد.

دستورالعمل نگارش و تدوین پایان‌نامه دانشگاه تهران به دو مقوله می‌پردازد، اول قالب و چگونگی صفحه‌آرایی پایان‌نامه، مانند اندازه و نوع قلم بخشهای مختلف، چینش فصلها، قالب مراجع و مواردی از این قبیل و دوم محتوای هر فصل پایان‌نامه. 
درصورت استفاده از این کلاس، نیازی نیست که دانشجو نگران مقوله اول باشد و پس از تایپ مطالب خود می‌تواند آنها را با لاتک و ابزار آن اجرا کند تا پایان‌نامه خود را با قالب دانشگاه داشته باشد. همچنین با خواندن این راهنما از ملزومات محتوایی هر فصل پایان‌نامه نیز مطلع خواهد شد.

در ادامهٔ  مقدمهٔ این راهنما، ابتدا چگونگی استفاده از کلاس پایان‌نامه و فایل‌های همراه آن را به صورت فنی شرح می‌دهیم و سپس مطالبی را در مورد ویژگی‌های محتوایی فصل ۱ پایان‌نامه (یعنی مقدمه) خواهیم آورد.
بقیهٔ فصل‌های این راهنما، تنها خصوصیات محتوایی فصول مختلف پایان‌نامه را شرح خواهند داد. نهایتاً جهت یادآوری، در پیوست‌ها مطالبی دربارهٔ آشنایی با دستورات لاتک، مدیریت مراجع در لاتک و چگونگی رسم جداول، نمودارها و الگوریتم‌ها آورده خواهند شد.

\section{چگونگی استفاده از کلاس پایان‌نامه}
کلیه فایل‌های لازم برای حروف‌چینی با کلاس فوق، داخل پوشه‌ای به نام
\lr{tehran-thesis}
قرار داده شده است. توجه داشته باشید که برای استفاده از این کلاس باید فونت‌های
\lr{IRLotusICEE}
و
\lr{IRTitr}
را داشته باشید (که همراه با این کلاس هست و نیاز به نصب نیست).
قلم‌های
\lr{IRLotusICEE}
مستخرج از قلم‌های استاندارد
\lr{IRLotus}
شورای عالی اطلاع‌رسانی%
\footnote{
قلم‌های استاندارد
\lr{IRFonts}
از شورای عالی اطلاع‌رسانی، منطبق بر آخرین نسخه استاندارد یونیکد، استاندارد ملی ۶۲۱۹ و استاندارد
\lr{Adobe Glyph Naming}
هستند.
}
هستند که توسط دکتر بابایی‌زاده اصلاحاتی روی آنها صورت پذیرفته است: تبدیل صفر توپر به صفر توخالی (جهت تمایز بیشتر با نقطه) و اضافه شدن
\textit{\textbf{حالت توپر و ایرانیک توأم}}،
که این موارد در قلم‌های شورای عالی اطلاع‌رسانی وجود ندارد.

\subsection{این همه فایل؟!}
\label{muchFiles}
از آنجایی که یک پایان‌نامه یا رساله، یک نوشته بلند محسوب می‌شود، لذا اگر همه تنظیمات و مطالب پایان‌نامه را داخل یک فایل قرار بدهیم، باعث شلوغی و سردرگمی می‌شود. به همین خاطر، قسمت‌های مختلف پایان‌نامه یا رساله  داخل فایل‌های جداگانه قرار گرفته است. مثلاً تنظیمات پایه‌ای کلاس داخل فایل
\lr{tehran-thesis.cls}، 
قسمت مشخصات فارسی پایان‌نامه داخل 
\lr{faTitle.tex}،
مطالب فصل اول داخل 
\lr{chapter1.tex}
و تنظیمات قابل تغییر توسط کاربر داخل 
\lr{commands.tex}،
قرار داده شده است.
\textbf{
	فایل اصلی این مجموعه، فایل
	\lr{main.tex}
	می‌باشد.
}
% یعنی بعد از تغییر فایل‌های دیگر، برای دیدن نتیجه تغییرات، باید این فایل را اجرا کرد. بقیه فایل‌ها به این فایل، کمک می‌کنند تا بتوانیم خروجی کار را ببینیم.
اگر به فایل 
\lr{main.tex}
دقت کنید، متوجه می‌شوید که قسمت‌های مختلف پایان‌نامه، توسط دستورهایی مانند 
\lr{input}
و
\lr{include}
به فایل اصلی، یعنی 
\lr{main.tex}
معرفی شده‌اند.
با توجه به ساختار محتوایی دستورالعمل، در فایل
\lr{main.tex}
فرض شده که پایان‌نامه یا رساله شما، از ۵ فصل و تعدادی پیوست تشکیل شده است. با اینحال، شما می‌توانید به راحتی فصل‌ها و پیوست‌ها را با صلاحدید اساتید راهنما، کم و زیاد کنید. این کار، بسیار ساده است. فرض کنید بخواهید یک فصل دیگر هم به پایان‌نامه اضافه کنید. برای این کار، کافی است یک فایل با نام دلخواه مثلاً 
\lr{chapter6}
و با پسوند 
\lr{.tex}
بسازید و آن را داخل پوشه 
\lr{tehran-thesis}
قرار دهید و سپس این فایل را با دستور 
\verb!\include{chapter6}!
داخل فایل
\lr{main.tex}
 فراخوانی کنید.

\subsection{از کجا شروع کنم؟}
قبل از هر چیز، باید یک توزیع تِک مناسب مانند تک‌لایو
\lr{(TeXLive)}
را روی سیستم خود نصب کنید. تک‌لایو  را می‌توانید از 
 \href{http://www.tug.org/texlive}{سایت رسمی آن}%
\LTRfootnote{\lr{\url{http://www.tug.org/texlive}}}
 دانلود کنید یا مستقیماً از مخازن توزیع لینوکس خود بگیرید (مثلاً در اوبونتو با دستور
\LRE{\verb!sudo apt install texlive-full!}).
برای نصب تک‌لایو و اجرای اسناد زی‌پرشین می‌توانید از
\href{http://parsilatex.com/site/shop/}{دی‌وی‌دی مجموعه پارسی‌لاتک}%
\LTRfootnote{\lr{\url{http://parsilatex.com/site/shop/}}}
و فایل راهنمای موجود در آن هم کمک بگیرید.

برای تایپ و پردازش اسناد لاتک باید از یک ویرایشگر مناسب استفاده کنید. ویرایشگرهای
\lr{TeXWroks},
\lr{TeXstudio},
\lr{Texmaker}
و
\lr{BiDiTeXmaker}
بدین منظور تولید شده‌اند. می‌توان ویرایش‌گر 
 \href{https://bitbucket.org/srazi/biditexmaker3}{\lr{BiDiTeXmaker}}%
 \LTRfootnote{\lr{\url{https://bitbucket.org/srazi/biditexmaker3}}}
را که بویژه برای کار با زی‌پرشین و مطالب دوجهته بهبود یافته است، بهینه‌ترین ویرایشگر لاتک برای کار با اسناد فارسی عنوان کرد.
 
حال اگر نوشتن \پ اولین تجربه شما از کار با لاتک است، توصیه می‌شود که یک‌بار به صورت اجمالی، کتاب «%
\href{http://www.tug.ctan.org/tex-archive/info/lshort/persian/lshort.pdf}{مقدمه‌ای نه چندان کوتاه بر
\lr{\LaTeXe}}%
\LTRfootnote{\lr{\url{http://www.tug.ctan.org/tex-archive/info/lshort/persian/lshort.pdf}\hfill}}»
ترجمه دکتر مهدی امیدعلی را مطالعه کنید. این کتاب، کتاب بسیار کاملی است که خیلی از نیازهای شما در ارتباط با حروف‌چینی را برطرف می‌کند.
اگر تک لایو کامل را داشته باشید، این کتاب را هم دارید. کافیست در خط فرمان دستور زیر را بزنید:
\begin{latin}
	\texttt{texdoc lshort-persian}
\end{latin}
اگر عجله دارید، برخی دستورات پایه‌ای مورد نیاز در پیوست \ref{app:latexIntro} بیان شده‌اند.
 
بعد از موارد گفته شده، فایل 
\lr{main.tex}
و
\lr{faTitle.tex}
را باز کنید و مشخصات پایان‌نامه خود مثل نام، نام خانوادگی، عنوان پایان‌نامه و ... را جایگزین مشخصات موجود در فایل
\lr{faTitle.tex}
 کنید. نیازی نیست نگران چینش این مشخصات در فایل پی‌دی‌اف خروجی باشید، زیرا کلاس 
\lr{tehran-thesis}
همه این کارها را بطور خودکار برای شما انجام می‌دهد. در ضمن، موقع تغییر دادن دستورهای داخل فایل
\lr{faTitle.tex}
 کاملاً دقت کنید؛ این دستورها، خیلی حساس هستند و ممکن است با یک تغییر کوچک، موقع اجرا، خطا بگیرید. برای دیدن خروجی کار، فایل 
\lr{faTitle.tex}
 را 
\lr{Save}
(نه 
\lr{Save As})
کنید و بعد به فایل 
\lr{main.tex}
برگشته و آن را اجرا کنید%
\footnote{
	البته فایلهای این مجموعه به گونه‌ای هستند که در
	\lr{TeXWorks} یا
	\lr{TeXstudio}
	بدون بازگشت به فایل اصلی، می‌توانید سند خود را اجرا کنید.
}.
 حال اگر می‌خواهید مشخصات انگلیسی \پ را هم عوض کنید، فایل 
\lr{enTitle.tex}
را باز کنید و مشخصات داخلش را تغییر دهید.
%\RTLfootnote{
%برای نوشتن پروژه کارشناسی، نیازی به وارد کردن مشخصات انگلیسی پروژه نیست. بنابراین، این مشخصات بطور خودکار، نادیده گرفته می‌شود.
%}
در اینجا هم برای دیدن خروجی باید این فایل را ذخیره کرده، بعد به فایل 
\lr{main.tex}
برگشته و آن را اجرا کرد.

برای راحتی بیشتر، کلاس 
\lr{tehran-thesis.cls}
طوری طراحی شده است که کافی است فقط  یک‌بار مشخصات \پ را (در فایل‌های
\lr{faTitle.tex}
و
\lr{enTitle.tex})
وارد کنید و هر جای دیگر که این مشخصات لازم باشند، به طور خودکار درج می‌شوند. با این حال، اگر مایل بودید، می‌توانید تنظیمات موجود را تغییر دهید؛ گرچه، در صورتیکه کاربر مبتدی هستید و یا با ساختار فایل‌های  
\lr{cls}
 آشنایی ندارید، بهتر است به فایل 
\lr{tehran-thesis.cls}
دست نزنید.

نکته دیگری که باید به آن توجه کنید این است که در قالب آماده شده، سه گزینه به نام‌های
\lr{bsc}،
\lr{msc}
و
\lr{phd}
برای نوشتن پروژه، پایان‌نامه و رساله، در نظر گرفته شده است. بنابراین اگر قصد تایپ پروژهٔ کارشناسی، پایان‌نامهٔ کارشناسی ارشد یا رسالهٔ دکتری را دارید، به ترتیب باید از گزینه‌های
\lr{bsc}،
\lr{msc}
و
\lr{phd}
در فایل 
\lr{main.tex}
استفاده کنید. با انتخاب هر کدام از این گزینه‌ها، تنظیمات مربوط به آنها به طور خودکار، اعمال می‌شود.


\subsection[مطالب پایان‌نامه را چطور بنویسم؟]
{مطالب \پ را چطور بنویسم؟}
\subsubsection{نوشتن فصل‌ها}
همان‌طور که در بخش \ref{muchFiles} گفته شد برای جلوگیری از شلوغی، قسمت‌های مختلف \پ از جمله فصل‌ها، در فایل‌های جداگانه‌ای قرار داده شده‌اند. 
مثلاً اگر می‌خواهید مطالب فصل ۱ را تایپ کنید، باید فایل‌های 
\lr{main.tex}
و
\lr{chapter1.tex}
را باز کرده و مطالب خود را جایگزین محتویات داخل 
\lr{chapter1.tex}
نمایید. دقت شود که در ابتدای برخی فایلها دستوراتی نوشته شده است و از شما خواسته شده که آن دستورات را حذف نکنید.

%توجه کنید که همان‌طور که قبلاً هم گفته شد، تنها فایل قابل اجرا، 
%\lr{main.tex}
%است. لذا برای دیدن حاصل (خروجی) فایل خود، باید  
%\lr{chapter1.tex}
%را ذخیره کرده و سپس فایل 
%\lr{main.tex}
%را اجرا کنید.

نکته بسیار مهمی که در اینجا باید گفته شود این است که سیستم \lr{\TeX}، محتویات یک فایل تِک را به ترتیب پردازش می‌کند.  بنابراین، اگر مثلاً  دو فصل اول خود را نوشته و خروجی آنها را دیده‌اید و مشغول تایپ مطالب فصل ۳ هستید، بهتر است
که دو دستور 
\verb!% !TeX root=../main.tex

\chapter{مقدمه و بیان مساله}
% دستور زیر باعث عدم‌نمایش شماره صفحه در اولین صفحه‌ی این فصل می‌شود.
%\thispagestyle{empty}
\section{مقدمه}
هر روزه برنامه‌های بیشتری تحت وب ساخته می‌شود که به صورت کاملا مستقل از هم مستقر می‌شوند و با استفاده از
\gls{RPC}
با هم ارتباط برقرار می‌کنند. پیدایش روش‌های کم‌هزینه مجازی‌سازی (برای مثال
\glspl{Container})
امکان به کارگیری تعداد زیادی سرویس ریزدانه را فراهم کرده‌اند که آن‌ها را میکروسرویس می‌نامند. شکستن برنامه‌ها به این شکل سودهای بسیاری خواهد داشت:‌ در این صورت مقیاس‌پذیری ساده می‌شود (هر سرویس به طور مستقل می‌تواند رشد کند)،‌ انعطاف‌پذیری بیشتری در تخصیص منابع و زمان ارائه می‌شود، امکان استفاده مجدد از کدها بیشتر می‌شود، امکان استفاده از روش‌های جدید تحمل خطا به وجود خواهد آمد، و امکان استفاده از سرویس‌های خارجی مانند
\lr{Amazon S3}
میسر می‌شود. در نتیجه این معماری توسط تعداد زیادی از شرکت‌های بزرگ و استارتاپ‌ها استفاده می‌شود (برای مثال
\lr{Uber} \cite{toddhoff2020}
و
\lr{Netflix} \cite{tonymauro2020})
و همینطور در مقیاس‌های بسیار بزرگی نیز استقرار یافته است (برای مثال برنامه شرکت
\lr{Uber} \cite{toddhoff2020}
از بیش از ۱۰۰۰ میکروسرویس تشکیل شده است).

در این تحقیق ما قصد داریم تا امکان درستی‌سنجی برنامه‌های مبتنی بر معماری میکروسرویس را بررسی کنیم و روشی برای انجام این کار ارائه دهیم.

\section{تاریخچه‌ای از موضوع تحقیق}
معماری میکروسرویس یک معماری بسیار جدید است که در چند سال اخیر به سبب رشد سریع برنامه‌های تحت وب شکل گرفته است. به همین سبب کارهای زیادی روی درستی‌سنجی برنامه‌های با این معماری صورت نگرفته است. در اینجا ما از بررسی روش‌های درستی‌سنجی سیستم‌های توزیع‌شده صرف نظر می‌کنیم و فقط روی درستی‌سنجی نرم‌افزارهای با معماری میکروسرویس تمرکز خواهیم کرد.

سیستم
ucheck \cite{panda2017verification}
یک سیستم برای درستی‌سنجی میکروسرویس‌ها است. این سیستم،‌ به صورت جدا از میکروسرویس‌ها مستقر می‌شود و با مانیتور کردن ترافیک شبکه و پیام‌های رد و بدل شده، سعی در بررسی درستی شرط‌های تعریف شده را دارد. این سیستم از استدلال
\lr{rely-guarantee} \cite{Jones83}
برای شرط‌ها استفاده می‌کند. به دلیل پیچیدگی پیاده‌سازی تحلیل ترافیک شبکه، این سیستم هیچ پیاده‌سازی در دسترسی ندارد. همینطوری استدلال
\lr{rely-guarantee}
برای بررسی دقیق سیستم‌های توزیع‌شده طراحی شده و نوشتن شرط‌های مناسب در آن، کاری بسیار دشوار است.

سیستم
Whip \cite{waye2017whip}
یک سیستم دیگر برای درستی‌سنجی میکروسرویس‌هاست. این سیستم هم مانند ucheck سعی می‌کند تا با تحلیل ترافیک شبکه عمل درستی‌سنجی را انجام دهد و به همین علت پیاده‌سازی مناسب استفاده عملیاتی ندارد. Whip از قراردادها برای توصیف شرط‌ها استفاده می‌کند و روشی برای توصیف قراردادهای سطح‌ بالا ارائه می‌دهد که از آن برای
\gls{BlameAssignment}
دقیق‌تر استفاده می‌کند.

\section{شرح مسئله تحقیق}
با بزرگ شدن اندازه برنامه و بیشتر شدن پیچیدگی‌های آن، اطمینان از درستی این برنامه‌های مبتنی بر معماری میکروسرویس برای گردانندگان آن سخت‌تر می‌شود. ویژگی‌های این برنامه‌ها، وابسته است به رفتار هر کدام از میکروسرویس‌های تشکیل دهنده آن، و تنظیمات مربوط به ارتباطات بین آن‌ها. فهمیدن برقراری یک شرط خاص در هنگام اجرای یک برنامه یک مساله غیربدیهی برای برنامه‌های کوچک است، اما برای برنامه بزرگی که از تعدادی زیاد میکروسرویس تشکیل شده است، بسیار سخت است. برای مثال، یک برنامه ساده را در نظر بگیرید که از سه سرویس تشکیل شده است: یک وب‌سرور، یک سرویس احراز هویت و یک پایگاه داده. اطمینان از این شرط که تنها کاربرانی که احراز هویت آن‌ها تایید شده است، می‌توانند پایگاه داده را به‌روزرسانی کنند، نیازمند دسترسی به وضعیت وب‌سرور (برای تعیین درخواستی که موجب به‌روزرسانی شده است) و سرویس احراز هویت (برای بررسی احراز هویت درخواست‌دهنده)‌ علاوه بر نیازمندی دسترسی به پایگاه داده است. حتی اگر تمامی میکروسرویس‌ها هم روش‌هایی را برای دسترسی به وضعیت‌شان ارائه کنند، باز هم نیاز داریم تا آن‌ها را با هم هماهنگ کنیم تا یک تصویر نامتناقض از کل برنامه به دست آوردیم. استفاده از چنین روش‌هایی در برنامه‌های بزرگ، بسیار ناکارآمد خواهد بود.

\section{تعریف موضوع تحقیق}
در قسمت‌های قبل، به نیاز به درستی‌سنجی برنامه‌های با معماری میکروسرویس و پیچیدگی‌های آن اشاره شد. در این تحقیق ما روی موضوع درستی‌سنجی نرم‌افزارهای با معماری میکروسرویس تمرکز می‌کنیم. در این تحقیق ما سعی می‌کنیم که در یک برنامه‌ بزرگ با تعداد سرویس‌های بسیار زیاد، از برقراری شرط‌های مهمی که توسط کاربر تعیین می‌شوند، اطمینان حاصل کنیم.

\section{اهداف و آرمان‌های کلی تحقیق}
در این تحقیق ما یک روش برای درستی‌سنجی برنامه‌های با معماری میکروسرویس ارائه می‌کنیم به گونه‌ای که مشکلات تحقیقات قبلی را نداشته باشد. در این تحقیق ما روی درستی‌سنجی ارتباط بین میکروسرویس‌ها تمرکز می‌کنیم چرا که پیچیدگی اصلی در در این معماری، ارتباطات زیاد بین تعداد زیاد سرویس است که روی شبکه و با استفاده از
\gls{RPC}
با هم صحبت می‌کنند. همینطور به دلیل اینکه این مسئله، یک مسئله کاربردی مهندسی نرم‌افزار است، وجود یک پیاده‌سازی مناسب برای محیط‌های عملیاتی بخش مهمی از تحقیق است. به همین دلیل،‌ ارائه یک پیاده‌سازی مناسب از اهداف اصلی این تحقیق است.

\section{روش انجام تحقیق}
با توجه به اهداف گفته شده، ما استفاده از قراردادها را برای نوشتن شرط‌ها و بررسی آن‌ها انتخاب کردیم. طراحی توسط قرارداد روش قدرتمندی است که با استفاده از آن به خوبی می‌توان شرط‌های لازم را برای
\glspl{RemoteProcedureCall}
نوشت. ما این شرط‌ها در زمان اجرای برنامه بررسی می‌کنیم و در صورت نقض شدن هر شرط، آن را به توسعه‌دهنده اطلاع می‌دهیم. در این اشکالات در کل سیستم پخش نمی‌شوند و توسعه‌دهنده در اولین فرصت متوجه آن‌ها می‌شود. همچنین در روش ما این امکان وجود دارد که بتوان روی ترتیب اجرا
\gls{RPC}ها
شرط نوشت. این موضوع کمک می‌کند که بسیار بهتر بتوان ارتباط میان میکروسرویس‌ها را درستی‌سنجی کرد.

سپس برای روش ارائه شده، یک پیاده‌سازی در زبان برنامه‌نویسی
Go \cite{golang}
و برای چارچوب
gRPC \cite{grpc}
ارائه می‌دهیم و سپس با استفاده از آن به درستی‌سنجی یک برنامه با معماری میکروسرویس می‌پردازیم و نشان می‌دهیم که چگونه در آن اشکالاتی را پیدا کردیم. در نهایت سربار روشمان را روی کارایی برنامه بررسی می‌کنیم و نشان می‌دهیم که روش ما سربار بسیار کمی خواهد داشت.

\section{ساختار پایان‌نامه}
!
و
\verb!% !TeX root=../main.tex
\chapter{مفاهیم اولیه و پیش‌زمینه}
%\thispagestyle{empty} 

\section{مقدمه}
در این فصل برای روشن‌شدن مفاهیم و اصطلاحات حول موضوع این پژوهش، به تعریف برخی مفاهیم اساسی که پیش‌نیاز درک این پروژه هستند می‌پردازیم.

\section{معماری میکروسرویس}
میکروسرویس، همان طور که از نام آن مشخص می‌شود، اساساً به سرویس‌های نرم‌افزاری مستقلی گفته می‌شود که کارکردهای تجاری خاصی را در یک اپلیکیشن نرم‌افزاری ارائه می‌کنند. این سرویس‌ها می‌توانند به صورت مستقل از هم نگهداری، نظارت و توزیع شوند.

\gls{ServiceOrientedArchitecture}
به اپلیکیشن‌ها امکان ارتباط با یکدیگر روی یک رایانه منفرد و یا در زمان توزیع اپلیکیشن‌ها روی چندین رایانه در یک شبکه را ارائه می‌کند. هر میکروسرویس ارتباط اندکی با سرویس‌های دیگر دارد. این سرویس‌ها خودکفا هستند و یک کارکرد منفرد (یا گروهی از کارکردهای مشترک) را ارائه می‌کنند.

معماری میکروسرویس‌ها به طور طبیعی در سازمان‌های بزرگ و پیچیده استفاده می‌شود که در آن‌ها چند تیم توسعه می‌توانند مستقل از هم برای ارائه یک کارکرد تجاری کار بکنند و یا اپلیکیشن‌ها ملزم به ارائه خدمات به یک حوزه تجاری باشند.

پیش از آن که بخواهیم به توضیح مسائلی که میکروسرویس برای حل آن‌ها ابداع شده بپردازیم، باید تاریخچه تکامل نرم‌افزار را به طور خلاصه مرور کنیم.

\subsection{تکامل رایانه‌ها}
زمانی که رایانه‌ها برای اولین بار در دهه 1940 میلادی ارائه شدند، نرم‌افزار به صورت پانچ شده درون سخت‌افزارهای بزرگ و گران‌قیمت مانند کارت‌های پانچ و نوارهای پانچ جاسازی شده بود. دستورالعمل‌ها به زبان ماشین باینری نوشته شده بودند. متعاقباً اگر لازم می‌شد تغییری پیاده‌سازی شود، یک کارشناس زبان باینری ماشین باید دستورالعمل‌های جدید را در اختیار ماشین قرار می‌داد. این امر یک فرایند بسیار گران‌قیمت بود.

سپس نسل دوم رایانه‌ها در دهه 1950 میلادی به عنوان نسخه بهبود یافته نسل اول رایانه‌ها معرفی شدند. این رایانه‌ها به برنامه‌هایی با زبان اسمبلی نیاز داشتند که در تراشه‌های سخت‌افزاری کوچک‌تری نوشته می‌شد. فلسفه طراحی این رایانه‌ها پیرامون این واقعیت شکل گرفته بود که یادگیری زبان نمادین اسمبلی بسیار راحت‌تر از آموختن کدهای باینری است. در نتیجه اسمبلرها معرفی شدند که کد ماشین را به زبان نمادین اسمبلی تبدیل می‌کردند.

در این مورد نیز هر زمان که تغییری مورد نیاز بود، فرد باید دستورالعمل‌ها را در سخت‌افزار وارد می‌کرد. در نتیجه نرم‌افزار و سخت‌افزار باید از همدیگر جدا می‌شدند.

و سپس نسل سوم رایانه‌ها در دهه 1960 میلادی عرضه شدند. این رایانه‌ها به کامپایلر و مفسر برای ترجمه زبان قابل فهم از سوی انسان به کد ماشین نیاز داشتند. این رایانه‌ها کوچک‌تر بودند و می‌بایست نرم‌افزارهای قابل تعامل از سوی کاربر روی آن‌ها نصب می‌شد. نسل سوم رایانه‌ها می‌توانستند چندین اپلیکیشن را همزمان اجرا کنند. زبان‌های برنامه‌نویسی مختلفی معرفی شدند که روی این رایانه‌ها نصب می‌شدند. سخت‌افزار همچنان گران‌قیمت بود؛ اما انعطاف‌پذیری حاصل از جداسازی نرم‌افزار از سخت‌افزار موجب ایجاد فرصت‌های بی‌شماری برای بهبود کارکرد آن‌ها شد.

در نهایت نسل چهارم رایانه‌ها در دهه 1970 میلادی معرفی شدند. این رایانه‌ها دستگاه‌های دستی بودند که می‌توانستند در کف دست جای بگیرند. در این مورد نیز اپلیکیشن‌های جدید معرفی شدند که می‌توانستند روی دستگاه‌ها بدون نیاز به خرید سخت‌افزار جدید نصب شوند. سهولت نصب و قابلیت نگهداری موجب شد که بسیاری از شرکت‌ها بتوانند ایده‌های فناورانه جدیدی را ابداع کنند.

یک طرح مشترک وجود دارد. رایانه‌ها به این دلیل تکامل یافتند که همواره نیاز به نگهداری و بهبود اپلیکیشن‌های نصب شده روی رایانه‌ها وجود داشته است.

\subsection{تکامل نرم‌افزار}
تکامل نرم‌افزار نیز چرخه مشابه را طی کرده است.
\begin{itemize}

\item
\textbf{
نسل اول (اواخر دهه 1970 میلادی):
}
مفاهیم شیءگرایی مانند وراثت، کپسوله‌سازی، و چندریختی معرفی شدند تا قابلیت استفاده مجدد از کد و قابلیت نگهداری فراهم شوند.

\item
\textbf{
نسل دوم (دهه 1990 میلادی):
}
اپلیکیشن‌ها با استفاده از معماری لایه‌بندی شده طراحی و پیاده‌سازی شدند. معماری لایه‌بندی شده برای کاهش در هم تنیدگی زیاد بین اجزای مختلف اپلیکیشن‌های نرم‌افزاری معرفی شد. در نتیجه تست نرم‌افزار و زمان برای تحویل نرم‌افزار کاهش یافت. اپلیکیشن‌ها همچنان یکپارچه بودند، یعنی یک واحد منفرد که همه کارکردهای اپلیکیشن وجود داشت در آن کپسوله‌سازی شده بود.

\item
\textbf{
نسل سوم (دهه 2000 میلادی):
}
در این زمان اپلیکیشن‌های با توزیع مبتنی بر سرویس معرفی شدند. روش طراحی مورد استفاده در این نسل شامل تعریف سرویس از راه دور بود که به اپلیکیشن‌ها امکان مقیاس‌بندی و توزیع روی ماشین‌های چندگانه را می‌داد.

\end{itemize}

پیش از معرفی میکروسرویس‌ها، اغلب اپلیکیشن‌ها از
\gls{MonolithicArchitecture}
استفاده می‌کردند. در ادامه مشکلات این معماری را با توجه به نیازمندی‌های جدید بررسی می‌کنیم.

\subsection{معماری یک‌پارچه}
یک برنامه با معماری یک‌پارچه از چندین لایه تشکیل می‌شود و هر لایه در کد نرم‌افزاری پیاده‌سازی می‌شد و از چندین کلاس و
\gls{Interface}
تشکیل یافته بود:

\begin{itemize}

\item
\textbf{
لایه داده:
}
این لایه برای ذخیره‌سازی داده‌ها در پایگاه داده و فایل‌ها پیاده‌سازی شده است. تنها مسئولیت این لایه ارائه داده‌ها از منابع داده‌ای مختلف است.

\item
\textbf{
لایه تجاری:
}
مسئولیت لایه تجاری، بازیابی داده‌ها از لایه داده و اجرای محاسبات است. لایه تجاری نمی‌داند که داده‌ها در فایل یا پایگاه داده یا در موارد دیگر هستند. لایه تجاری به لایه داده‌ها وابسته است.

\item
\textbf{
لایه سرویس:
}

لایه سرویس روی لایه تجاری قرار می‌گیرد. این لایه یک پوشش برای لایه تجاری ایجاد می‌کند که شامل امنیت/گزارش‌گیری/وساطت برای فراخوانی کارکردها است. لایه سرویس به لایه تجاری وابسته است.

\item
\textbf{
لایه رابط کاربری:
}
لایه اینترفیس یا رابط کاربری شامل کدهایی است که در لایه میزبانی مورد ارجاع قرار می‌گیرند و به منظور ایجاد امکان تعامل با اپلیکیشن برای کاربران طراحی شده است.

\end{itemize}

این طراحی به توسعه‌دهندگان امکان می‌دهد که روی یک کارکرد خاص متمرکز شوند، ویژگی را تست کنند، و یا اپلیکیشن را با استفاده از کنترل معکوس از طریق
\glspl{DependencyInjection}
و میزبانی یک اپلیکیشن روی ماشین‌های متعدد تجزیه کنند.

طراحی یکپارچه لایه‌بندی شده مزایای زیادی دارد؛ اما نواقصی نیز دارد. در ادامه فهرستی از مشکلات رایج این نوع معماری را مورد اشاره قرار داده‌ایم:
\begin{enumerate}

\item
این طراحی برای مقیاس‌بندی و نگهداری اپلیکیشن سرراست نیست. این طراحی زمان مورد نیاز برای قرار دادن کارکردهای جدید در اختیار کاربر را افزایش داده است، چون چرخه توسعه زمان بیشتری طول می‌کشد.

\item
از آنجا که کل اپلیکیشن به صورت یک پردازش منفرد میزبانی می‌شود، هر بار که لازم باشد یک به‌روزرسانی اجرا شود، کل اپلیکیشن باید متوقف شود و سپس نسخه جدیدی از اپلیکیشن باید توزیع شود.

\item
برای ایجاد تعادل در بار کاری، کل اپلیکیشن نرم‌افزاری روی چند ماشین توزیع می‌شود. به علاوه، امکان توزیع کارکردهای خاص روی سرورهای چندگانه برای متوازن‌سازی بار وجود ندارد.

\item
طراحی اپلیکیشن پیچیده است، چون همه ویژگی‌ها در یک اپلیکیشن یکپارچه منفرد پیاده‌سازی شده‌اند.

\item
زمانی که تعداد اپلیکیشن‌ها در سازمان افزایش می‌یابد، توزیع اپلیکیشن‌های یکپارچه نیازمند اطلاع‌رسانی و هماهنگی با همه تیم‌های توسعه ویژگی‌های جدید است. این امر موجب افزایش زمان مورد نیاز برای تست و توزیع اپلیکیشن می‌شود.

\item
بدین ترتیب در طراحی یک
\gls{SinglePointOfFailure}
ایجاد می‌شود که یک خطای غیر قابل بازیابی منفرد نمی‌تواند پردازشی را که اپلیکیشن روی آن میزبانی شده است متوقف کند.

\item
این معماری اپلیکیشن را وادار می‌کند که در یک مجموعه فناوری منفرد پیاده‌سازی شود.

\item
در مواردی که زمان تحویل طولانی‌تر شود، در طی زمان نیازمند پول بیشتری برای توسعه و نگهداری اپلیکیشن خواهد بود.

\item
از آنجا که همه کد درون یک اپلیکیشن منفرد قرار دارد، نگهداری کد پس از مدتی به سرعت دشوار می‌شود.

\end{enumerate}

\subsection{معماری میکروسرویس}

به طور خلاصه یک مفهوم وجود دارد و آن این است که ما همواره ملزم هستیم نرم‌افزار را نگهداری و به‌روزرسانی کنیم. باید فرایند بهبود اپلیکیشن‌ها را آسان‌تر و مقدار زمان مورد نیاز برای ارائه نسخه‌های جدید اپلیکیشن‌ها را کوتاه‌تر کنیم.

میکروسرویس‌ها برای حل مسائل اشاره شده فوق معرفی شده‌اند. معماری میکروسرویس یک بهینه‌سازی در زمینه معماری فوق‌الذکر محسوب می‌شود. در این معماری هر کارکرد تجاری به صورت یک سرویس ارائه می‌شود. هر سرویس می‌تواند به صورت مستقل از سرویس‌های دیگر میزبانی و توزیع شود.

به عنوان مزیت‌های استفاده از این معماری، می‌توان به موارد زیر اشاره کرد:
\begin{itemize}

\item
همه سرویس‌ها می‌توانند حتی زمانی که سرویس‌ها روی ماشین‌های مختلف هستند، با همدیگر ارتباط داشته باشند. این وضعیت در ادامه امکان پیاده‌سازی کارکردهای جدید در سرویس‌ها را فراهم می‌سازد.

\item
میکروسرویس‌ها، سازمان‌ها را تشویق می‌کنند که از فرایند توزیع و تحویل پیوسته خودکار پیروی کنند.

\item
اپلیکیشن‌ها در نهایت بسیار پایدارتر می‌شوند، چون هر ویژگی می‌تواند به صورت مستقلانه تست و توزیع شود.

\item
از آنجا که هر سرویس روی پردازش مجزایی میزبانی می‌شود، اگر یک سرویس به نقطه تنگنای اپلیکیشن تبدیل شود و به منابع زیادی نیاز داشته باشد، در این صورت می‌توان آن را بدون هیچ گونه تأثیر سوء روی سرویس‌های دیگر، به ماشین دیگری انتقال داد.

\item
زمانی که کاربران بیشتری شروع به استفاده از یک ویژگی اپلیکیشن بکنند، آن سرویس می‌تواند با توزیع روی رایانه‌های قدرتمندتر یا از طریق استفاده از کش بدون این که روی سرویس‌های دیگر هیچ تأثیری بگذارد، مقیاس‌بندی شود.

\item
این معماری پایداری اپلیکیشن را نیز افزایش می‌دهد، چون هر سرویس می‌تواند به صورت مستقلانه ساخته، تست، توزیع و استفاده شود.

\item
کد اپلیکیشن می‌تواند به سادگی نگهداری شود و پردازش‌ها می‌توانند به صورت مجزا تحت نظارت قرار بگیرند.

\item
توسعه‌دهندگان اختصاصی می‌توانند سرویس‌ها را به صورت مستقل از هم پیاده‌سازی کرده و این سرویس‌ها را بدون تأثیرگذاری روی سرویس‌های دیگر انتشار دهند.

\item
بدین ترتیب نقطه شکست منفرد نیز از بین می‌رود، زیرا یک سرویس می‌تواند بدون تأثیرگذاری روی ویژگی‌های دیگری که اپلیکیشن نرم‌افزاری ارائه می‌کند، متوقف شود.

\item
در نتیجه این طراحی زمان مورد نیاز برای تحویل نسخه‌های جدید را کاهش می‌دهد و بنابراین هزینه را در طی زمان کاهش می‌دهد.

\item
قابلیت استفاده مجدد از کد افزایش می‌یابد، زیرا یک ویژگی به صورت سرویس میزبانی شده است و امکان استفاده چند سرویس از یک ویژگی به جای پیاده‌سازی مجدد کد در هر مورد وجود دارد.

\item
معماری مبتنی بر سرویس امکان استفاده از مجموعه متنوعی از فناوری برای رفع نیازها وجود دارد. به عنوان نمونه بسته‌های تحلیل داده زبان R یا پایتون می‌توانند به صورت مجزا توزیع و میزبانی شوند و همزمان می‌توان از 
\lr{C\#.Net}
برای پیاده‌سازی سرویس‌ها استفاده کرد. به علاوه می‌توان از NodeJS در سمت سرور استفاده کرد و AngularJs و ReactJs نیز برای پیاده‌سازی رابط کاربری مورد استفاده قرار گیرند. هر ویژگی تجاری می‌تواند با استفاده از مجموعه مختلفی و از طریق تیم متفاوتی، مستقل از ویژگی‌های دیگر پیاده‌سازی شود.

\end{itemize}


\section{درستی‌سنجی نرم‌افزار}
درستی‌سنجی نرم‌افزار یکی از بخش‌های مهندسی نرم‌افزار است که هدف آن اطمینان حاصل کردن از این است که نرم‌افزار تمام نیازمندی‌های تعریف شده را برآورده کند. درستی‌سنجی نرم‌افزار می‌توان به دو صورت ایستا و پویا انجام داد. در درستی‌سنجی ایستا قبل از اجرای برنامه، بررسی صورت می‌گیرد و در درستی‌سنجی پویا، بررسی برنامه در هنگام اجرای آن صورت می‌گیرد. 

هدف درستی‌سنجی پویا پیدا کردن اشکالات برنامه در زودترین زمان ممکن است و قبل اینکه آن‌ها در تمام برنامه منتشر شوند. در ادامه یکی از قوی‌ترین روش‌های درستی‌سنجی پویا، به نام طراحی بر اساس قرارداد را بررسی می‌کنیم.

\subsection{طراحی بر اساس قرارداد}

قرارداد در واقع تشبیهی از زندگی روزمره است. قرارداد چیزی است که همه‌ی ما با آن آشنایی داریم. منظور این است که زمانی که با یک شئ تعامل می‌کنید، شما به عنوان فراخواننده، و شئ به عنوان آن‌چه فراخوانده می‌شود، روی یک قرارداد توافق می‌کنید. این شئ است که مفاد قراداد را تعیین می‌کند و شما یا باید آن را بپذیرید یا از آن استفاده نکنید. هر تابع از یک شئ تضمین می‌کند اگر شما به عنوان فراخواننده به بخش‌های مربوط به خود از قرداد پایبند باشید (اگر 
\glspl{Precondition}
مورد نیاز آن شئ را برآورده کنید) کار خود را انجام دهد. مسئله‌ی مهم این است که مانند زندگی روزمرّه، قرارداد چیزی دوطرفه است. فقط این نیست که شئ کاری را انجام دهد و من به عنوان فراخواننده وظایفم را انجام ندهم. این‌طور هم نیست که من به عنوان فراخواننده وظایفم را انجام دهم و شئ به دلخواه خود عمل کند - مثلاً اگر من یک اتاق را اجاره کنم، کلیدش را می‌گیرم و می‌توانم در آن زندگی کنم، اما به شرط اینکه اجاره‌اش را پرداخت کنم - بنابراین کاملاً دو طرفه است، فکر کردن به آن این‌گونه است.


پیش و پس‌شرط‌ها عبارات
\gls{Boolean}
هستند. هر تابع لیستی از پیش‌شرط‌ها (لیستی از عبارات که حاصل آن یک مقدار بولین است) را دارد. حاصل این لیست پیش‌شرط‌ها هنگام فراخوانی باید مقداری درست باشد. لیست دیگری هم از
\glspl{Postcondition}
 وجود دارد که آن هم لیستی از عبارات بولین است. شئ تضمین می‌کند اگر من پیش‌شرط‌ها را رعایت کنم، حاصل پس‌شرط‌ها هم پس از فراخوانی یک مقدار درست خواهد بود. فکر می‌کنم یک مثال در این‌جا مفید باشد. فرض کنید که یک تابع برای محاسبه‌ی ریشه‌ی دوم یک عدد اعشاری نوع Double داریم. بدیهی است که یک پیش‌شرط این است که پارامتر ورودی باید بزرگ‌تر یا مساوی صفر باشد. پس‌شرط هم این است که اگر نتیجه‌ی برگردانده‌ شده توسط تابع را در خودش ضرب کنم، همان مقدار ورودی حاصل شود. البته در واقع هیچ‌وقت دقیقاً همان عدد حاصل نخواهد شد. بنابراین اگر پس‌شرط را به صورت رسمی بیان شود باید گفت تفاوت میان نتیجه و عدد مورد نظر باید کم‌تر از یک عدد مشخص باشد. بنابراین شما دقتی برای محاسبات تابع تعیین می‌کنید. و این یکی از ارزش‌های مشخص کردن پیش و پس‌شرط‌ها به صورت رسمی را نشان می‌دهد، چون دقت عملیات را به صراحت در امضای تابع بیان می‌کند که در بسیاری از موارد مفید است.
بیشتر اوقات پس‌شرط‌های 
\gls{Functional}
را بررسی می‌کنید. به طور خاص این‌که مقدار محاسبه شده برخی شرایط را ارضا می‌کند اما گاهی اوقات هم افراد روی شرایط 
\gls{NonFunctional}
کار می‌کنند، مثلاً تعداد چرخه‌های پردازنده که این عملیات به طول می‌انجامد،‌ اما عموماً این پیش و پس‌شرط‌ها عملیاتی هستند.

\singlespacing
\begin{figure}
	\begin{LTR}
		\lstinputlisting[language=JAVA, caption={نمونه قرارداد یافتن ریشه دوم در زبان جاوا}, label={code:javaSqrt}]{sqrt.java}
	\end{LTR}
\end{figure}
\doublespacing

مهم است که بدانیم و مشخص شود پیش و پس‌شرط‌ها بر روی چه داده‌هایی می‌توانند اعمال شوند. بیایید با پیش‌شرط‌ها شروع کنیم. واضح است که در این مورد پارامترها هستند. در مثال ریشه‌ی دوم این مسئله واضح است که می‌گویید این پارامتر نباید 
\lr{\texttt{Null}}
باشد، یا باید در این بازه قرار بگیرد یا از این قبیل. این‌ها مثال‌های متداولی هستند که عموماً با 
\glspl{Assertion}
بررسی می‌شوند و کد آن‌ها را روی پارامترها اعمال می‌کند. دسته‌ی دیگرِ داده‌ها که پیش‌شرط‌ها می‌توانند روی آن اعمال شوند، وضعیت قابل مشاهده یا وضعیت عمومی شئ است. همان‌طور که در ابتدا گفتیم، موضوع طراحی بر اساس قرداد، واسط‌ها هستند، نه پیاده‌سازی. بنابراین تمام وضعیتی که از طریق واسط قابل دسترسی است، می‌تواند در پیش‌شرط‌ها شرکت داشته باشد. اگر به مثال مجموعه برگردیم، فرض کنید یک تابع 
\lr{\texttt{size()}}
داریم که تعداد عناصر درون مجموعه را از طریق واسط آن در دسترس قرار می‌دهد و تابع دیگری هم به نام
\lr{\texttt{getFirst()}}
داریم و فرض کنیم که قرارداد این مجموعه می‌گوید که در صورتی که مجموعه خالی است
\lr{\texttt{getFirst()}}
نباید فراخوانی شود بنابراین پیش‌شرط آن، به صورت
\lr{\texttt{size() > 0}}
تعریف می‌شود. این تنها راه انجام دادن این کار نیست، دلایل خوبی برای نداشتن چنین پیش‌شرطی وجود دارد. با این حال این کار امکان‌پذیر است. یعنی هیچ‌کس مجاز نیست وقتی مجموعه خالی است تابع
\lr{\texttt{getFirst()}}
را فراخوانی کند. فرض کنید که این مجموعه یک
\gls{Stack}
است و یک تابع
\lr{\texttt{pop()}}
دارد. این که تصور کنیم فراخوانی
\lr{\texttt{pop()}}
در صورتی که پشته خالی باشد مجاز نیست، خیلی متداول است. بنابراین دسترسی به وضعیت داخلی یک شیء از طریق واسط عمومی آن در نوشتن پیش‌شرط‌ها مجاز است. اما این امکان وجود ندارد که در پیش‌شرط‌ها به وضعیت داخلی که در واسط عمومی وجود ندارد ارجاع داشته باشیم. چون پیش‌شرط‌ها جزئی از واسط هستند، و واسط نباید اطلاعی از پیاده‌سازی داشته باشد و باید از آن مستقل باشد.
در مورد پس‌شرط‌ها مسائل کمی پیچیده‌تر هستند. در اینجا هم می‌توان به پارامترها ارجاع داشت. در مثالی که زدیم (مثال یافتن ریشه‌ی دوم یک عدد)، نتیجه‌ی محاسبات (اگر در خودش ضرب شود) باید تقریباً با پارامتر ورودی برابر باشد. بنابراین می‌توانید به حاصل یک تابع نیز دسترسی پیدا کنید، که بدیهی است ضروری است. همچنین می‌توانید به وضعیت شئ قبل و بعد از فراخوانی عملیات دسترسی پیدا کنید. فرض کنید در مثال مجموعه یک تابع
\lr{\texttt{add()}}
داشته باشیم که چیزی به مجموعه اضافه می‌کند. یکی از پس‌شرط‌ها این خواهد بود که اندازه‌ی مجموعه پس از عملیات یکی بیشتر از اندازه‌ی آن قبل از انجام عملیات است. بنابراین این یکی از چیزهایی است که به پیچیدگی‌های پس‌شرط‌ها اضافه می‌کند و در مورد پیش‌شرط‌ها به آن‌ها احتیاجی نیست.
!
را در فایل 
\lr{main.tex}،
غیرفعال%
\footnote{
برای غیرفعال کردن یک دستور، کافی است در ابتدای آن، علامت درصد انگلیسی (\%) بگذارید.
}
 کنید. در غیر این صورت، ابتدا مطالب دو فصل اول پردازش شده و سپس مطالب فصل ۳ پردازش می‌شود که این کار باعث طولانی شدن زمان پردازش می‌گردد. هر زمان که خروجی کل \پ را خواستید، تمام فصل‌ها را دوباره در
\lr{main.tex}
فعال نمائید.
بدیهتاً لازم نیست فصل‌های \پ را به ترتیب تایپ کنید. مثلاً می‌توانید ابتدا مطالب فصل ۳ را تایپ نموده و سپس مطالب فصل ۱ را تایپ کنید. 
\subsubsection{مراجع}
برای وارد کردن مراجع \پ کافی است فایل 
\lr{MyReferences.bib}
را باز کرده و مراجع خود را به شکل اقلام نمونهٔ داخل آن، وارد کنید.  سپس از \lr{bibtex} برای تولید مراجع با قالب مناسب استفاده نمائید. برای توضیحات بیشتر بخش \ref{Sec:Ref} از پیوست \ref{app:latexIntro} و نیز پیوست \ref{app:refMan} را ببینید.

\subsubsection{واژه‌نامه فارسی به انگلیسی و برعکس}
برای وارد کردن معادل فارسی اصطلاحات لاتین در متن و تهیه فهرست واژه‌نامه از آنها، از بستهٔ
\lr{glossaries}
و نرم‌افزار
\lr{xindy}
استفاده می‌شود. بدین منظور کافی است اصطلاحات لاتین و ترجمهٔ آنها را در فایل
\lr{words.tex}
وارد کرده و هر جای متن که خواستید با دستورات
\verb|gls{label}|
یا \verb|glspl{label}|
معادل فارسی مفرد یا جمع یک اصطلاح را بیاورید.

مثلا در اینجا، واژهٔ
«\gls{Action}»
برای بار اول و دوباره
«\gls{Action}»
برای بار دوم در متن ظاهر شده است.
جهت توضیحات بیشتر به پیوست
\ref{app:refMan}
مراجعه کنید.
\subsubsection{نمایه}
برای وارد کردن نمایه، باید از 
\lr{xindy}
استفاده کنید. 
%زیرا 
%\lr{MakeIndex}
%با حروف «گ»، «چ»، «پ»، «ژ» و «ک» مشکل دارد و ترتیب الفبایی این حروف را رعایت نمی‌کند. همچنین، فاصله بین هر گروه از کلمات در 
%\lr{MakeIndex}،
%به درستی رعایت نمی‌شود که باعث زشت شدن حروف‌چینی این قسمت می‌شود. 
راهنمای چگونگی کار با 
\lr{xindy} 
را می‌توانید در ویکی پارسی‌لاتک و یا مثالهای موجود در دی‌وی‌دی «مجموعه پارسی‌لاتک»، پیدا کنید.

\subsection{اگر سوالی داشتم، از کی بپرسم؟}
برای پرسیدن سوال‌های خود موقع حروف‌چینی با زی‌پرشین، می‌توانید به
\href{http://qa.parsilatex.com}{سایت پرسش و پاسخ پارسی‌لاتک}%
\LTRfootnote{http://qa.parsilatex.com}
یا
\href{http://forum.parsilatex.com}{بایگانی تالارگفتگوی قدیمی پارسی‌لاتک}%
\LTRfootnote{http://forum.parsilatex.com}
مراجعه کنید. شما هم می‌توانید روزی به سوال‌های دیگران در اینترنت جواب دهید.
بستهٔ زی‌پرشین و بسیاری از بسته‌های مرتبط با آن مانند
\lr{bidi} و
\lr{Persian-bib}،
مجموعه پارسی‌لاتک، مثالهای مختلف موجود در آن، قالب پایان‌نامه دانشگاههای مختلف و سایت پارسی‌لاتک همه به صورت داوطلبانه توسط افراد گروه پارسی‌لاتک و گروه
\lr{Persian TeX}
و بدون هیچ کمک مالی انجام شده‌اند. کار اصلی نوشتن و توسعه زی‌پرشین توسط آقای وفا خلیقی انجام شده است که این کار بزرگ را به انجام رساندند.
اگر مایل به کمک به گروه پارسی‌لاتک هستید به سایت این گروه مراجعه فرمایید:
\begin{center}
	\url{http://www.parsilatex.com}
\end{center}

\section{محتویات فصل اول یک پایان‌نامه}
فصل اول یک پایان‌نامه باید به مقدمه یا کلیات تحقیق بپردازد.
هدف از فصل مقدمه%
\LTRfootnote{Introduction}،
شرح مختصر مسأله تحقیق، اهمیت و انگیزه محقق از پرداختن به آن موضوع، بهمراه اشاره‌ای کوتاه به روش و مراحل تحقیق است. مقدمه، اولین فصل از ساختار اصلی \پ بوده و زمینه اطلاعاتی لازم را برای خواننده فراهم می‌آورد. در طول مقدمه باید سعی شود موضوع تحقیق با زبانی روشن، ساده و بطور عمیق و هدفمند به خواننده معرفی شود. این فصل باید خواننده را مجذوب و اهمیت موضوع تحقیق را آشکار سازد. در مقدمه باید با ارائهٔ سوابق، شواهد تحقیقی و اطلاعات موجود (با ذکر منبع) با روشی منظم، منطقی و هدف‌دار، خواننده را جهت داد و به سوی راه حل مورد نظر هدایت کرد. مقدمه مناسب‌ترین جا برای ارائهٔ اختصارات و بعضی توضیحات کلی است، توضیحاتی که شاید نتوان در مباحث دیگر آنها را شرح داد.

مقدمه، یکی از ارکان اساسی و اصلی پایان نامه است که مهمترین قسمت‌های آن عبارتند از: 

\subsection{عنوان تحقیق} 
باید شناختی دقیق و روشن از حوزهٔ موضوع تحقیق را عرضه دارد و خالی از هرگونه ابهام و پیچیدگی باشد.

\subsection{مسأله تحقیق}
وظیفه اصلی مقدمه بیان این مطلب به خواننده است که چرا انجام تحقیق را به عهده گرفته‌اید. اگر دلیل شما برای انجام این کار پاسخگویی به سؤال مورد علاقه‌تان است، با مشکل زیادی روبه‌رو نخواهید بود. یکی از بهترین روش‌ها برای نوشتن مقدمهٔ یک پایان‌نامه، طرح پرسش یا پرسش‌هایی مهم و اساسی است که کار تحقیقاتی شما از آغاز تا پایان قصد پاسخ دادن به آن را دارد. گاهی می‌توانید ابتدا اهمیت موضوع را بیان و سپس پرسش خود را در آن موضوع مطرح کنید.

\subsection{تاریخچه‌ای از موضوع تحقیق}
به طور کلی تشریح روندهای تحقیقاتی در محدودهٔ مورد مطالعه، مستلزم ارجاع به کارهای دیگران است. بعضی از نویسندگان برای کارهای دیگران هیچ اعتباری قائل نمی‌شوند و در مقابل، بعضی دیگر از نویسندگان در توصیف کارهای دیگران، بسیار زیاده‌روی می‌کنند. اکثر مواقع، ارجاع به مقالات دو سال قبل از کارتان، بهتر از نوشتن سطرهای مرجع است. در این قسمت باید به طور مختصر به نظرات و تحقیقات مربوط به موضوع و یا مسائل و مشکلات حل نشده در این حوزه و همچنین توجه و علاقه جامعه به این موضوع، اشاره شود.

\subsection{تعریف موضوع تحقیق}
در این قسمت محقق، موضوع مورد علاقه و یا نیاز احساس شدهٔ خود را در حوزه تحقیق بیان می‌دارد و عوامل موجود در موقعیت را تعریف و تعیین می‌کند.

\subsection{هدف یا هدف‌های کلی و آرمانی تحقیق}
این قسمت باید با جملات مثبت و کلی طرح شود و از طولانی شدن مطالب پرهیز شود.

\subsection{روش انجام تحقیق}
در این قسمت، پژوهشگر روش کاری خود را بیان می‌دارد و شیوه‌های گوناگونی را که در گردآوری مطالب خود بکار برده، ذکر می‌کند. همچنین اگر روش آماری خاصی را در تهیه و تدوین اطلاعات به کار برده است، آن شیوه را نیز اینجا بیان می‌کند.

\subsection{نوآوری، اهمیت و ارزش تحقیق}
در این قسمت، در مورد نوآوری علمی و عملی تحقیق که محقق به آن دست خواهد یافت، بحث می‌شود. ممکن است لازم باشد تا برخی نمودارهای خلاصه در این بخش استفاده شوند. به عنوان مثال، نموداری از مقاله
\cite{kim2016integrated}
در شکل
\ref{fig:sampleDiagram}
آمده است.
\begin{figure}[ht]
	\centerline{\includegraphics[width=0.8\textwidth]{journal-of-cancer_sample-result}}
	\caption{یک نمونه نمودار خلاصه برای نمایش نوآوری در نتایج
		%\cite{kim2016integrated}
	}
	\label{fig:sampleDiagram}
\end{figure}\\
طبیعتاً به صلاحدید نگارنده، شکل‌ها و نمودار‌ها می توانند در بخش های مختلف، خصوصا فصل
\ref{chap:results}
مورد استفاده قرار گیرند.

\subsection{تعریف واژه‌ها (اختیاری)}
در این قسمت محقق باید واژه‌هایی را که ممکن است برای خواننده آشنا نباشد، تعریف کند.

\subsection{خلاصه فصل‌ها}
در آخرین قسمتِ فصل اول پایان‌نامه، خلاصه‌ای اشاره‌وار از فصل‌های آتی آورده می‌شود تا خواننده بتواند تصویری واضح از دیگر قسمت‌های پایان‌نامه در ذهن خود ترسیم کند.

\section{جمع‌بندی}
در این فصل به دو مقولهٔ نحوه استفاده از قالب \پ دانشگاه تهران و نیز ویژگی‌هایی که محتویات فصل اول پایان‌نامه (یعنی مقدمه) باید داشته باشند، پرداخته شد. با توجه به اینکه این راهنما نحوه استفاده از قالب را شرح داده، ملزومات محتوایی هر فصل پایان‌نامه را توضیح می‌دهد و در پیوست‌ها نیز نحوهٔ کار با لاتک را یادآوری خواهد کرد، بنابراین مطالعهٔ کامل آن مقداری وقت شما را خواهد گرفت؛ اما مطمئن باشید از اتلاف وقت شما در ادامه کارتان تا حد زیادی جلوگیری خواهد کرد. در نوشتن متن حاضر سعی شده است علاوه بر ایجاد یک قالب لاتک برای پایان‌نامه‌های دانشگاه تهران، نکات محتوایی هر فصل نیز گوشزد گردد. طبیعتاً برای نگارش پایان‌نامهٔ خود می‌بایست مطالب تمام فصل‌ها را خودتان بازنویسی کنید.

در ادامهٔ این راهنما، تنها فصل‌هایی که یک پایان‌نامه باید داشته باشد و نیز خصوصیات یا ساختاری که محتویات هر فصل باید از آنها برخوردار باشد%
\footnote{از روی فایل «تمپلیت نگارش و تدوین پایان‌نامه \cite{UTThesisGuide}»}،
آورده می‌شوند. نهایتاً  در پیوست‌ها، مطالبی در باب یادآوری دستورات لاتک، نحوه نوشتن فرمول‌ها، تعاریف، قضایا، مثال‌ها، درج تصاویر، نمودارها، جداول و الگوریتم‌ها و نیز مدیریت مراجع، آمده است.

همچنین توصیه اکید دارم که رفع خطاهایی که احتمالاً با آنها مواجه می‌شوید را به آخر موکول نفرمایید و به محض برخورد با خطا، آن را اشکال‌زدایی و برطرف نمائید.
