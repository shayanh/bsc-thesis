% !TeX root=../main.tex
\chapter{بحث و نتیجه‌گیری}
%\thispagestyle{empty} 
\section{جمع‌بندی}
در این تحقیق در گام نخست با زمینه تحقیق، اهداف پیگیری‌شده و کارهایی که سابقا در این زمینه انجام شده‌است آشنا شدیم. سپس مفاهیم پایه‌ای مربوط به تحقیق عنوان و بررسی شد. در ادامه راه روشی ارائه شد و پیاده‌سازی‌ و کتابخانه توسعه داده‌شده، معرفی شدند. در نهایت روش درستی‌سنجی پیشنهاد شده روی یک پروژه واقعی پیاده شد و دیدیم که چگونه توانستیم چندین اشکال در این پروژه پیدا کنیم. همینطور سربار روش ارائه شده را نیز از نظر زمانی و حافظه بررسی کردیم و دیدیم که این سربار بسیار ناچیز است.

\section{نتیجه‌گیری}
با توجه به نتایج به دست آمده، دیدیم که روش طراحی بر اساس قرارداد و درستی‌سنجی پویا روش مناسبی برای درستی‌سنجی نرم‌افزارهای با معماری میکروسرویس است. چرا که با استفاده از این روش، ارتباط بین میکروسرویس‌ و انتظارات هر سرویس به صورت دقیق بررسی می‌شود و در صورت نقض شدن هر کدام از انتظارات، توسعه‌دهنده به سرعت متوجه خواهد شد.

\section{دست‌آورد‌ها}
ما در این تحقیق توانستیم از میان روش‌های موجود، روشی مناسب را برای درستی‌سنجی میکروسرویس‌ها انتخاب کنیم و آن را با توجه به این نیازمندی، بهینه کنیم. سپس یک پیاده‌سازی برای این روش ارائه دادیم و با درستی‌سنجی یک پروژه واقعی، کارایی آن را در عمل بررسی کردیم. یکی از دست‌آوردهای اصلی این تحقیق، امکان نوشتن شرط روی ترتیب اجرای فراخوانی‌های از راه دور است که به درستی‌سنجی ارتباط بین میکروسرویس‌ها کمک شایانی خواهد کرد.

\section{محدودیت‌ها}
در حال حاضر، محدودیت اصلی کتابخانه نوشته شده، وابسته بودن آن به تکنولوژی‌های انتخاب شده است و اگر سرویسی از تکنولوژی‌های دیگری استفاده کند، نمی‌توان از این کتابخانه استفاده کرد. همینطور در مقیاس‌های بسیار بزرگ ممکن است سربار زمانی زیاد شود که با توجه به محدودیت امکانات، این مورد را نتوانستیم بررسی کنیم.

\section{پیشنهادها}
برای ادامه این تحقیق، خوب است امکان این در نظر گرفته شود که سرویس‌هایی که با تکنولوژی‌های مختلف نوشته می‌شوند، بتوانند از کتابخانه ارائه شده استفاده کنند. همین‌طور واسط کتابخانه معرفی شده بسیار جای بهبود دارد و می‌توان قابلیت‌های خیلی بیشتری به آن اضافه کرد. همچنین می‌توان روی تحلیل رسمی روش ارائه شده کار کرد و آن را دقیق ارائه داد.
